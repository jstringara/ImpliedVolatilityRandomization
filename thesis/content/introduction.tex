\section{Introduction}
% \label{sec:introduction}
% This document is intended to be both an example of the Polimi \LaTeX{} template for Master Theses in article format,
% as well as a short introduction to its use. It is not intended to be a general introduction to \LaTeX{} itself,
% and the reader is assumed to be familiar with the basics of creating and compiling \LaTeX{} documents (see \cite{oetiker1995not, kottwitz2015latex}). 
% \\
% The cover page of the thesis in article format must contain all the relevant information:
% title of the thesis, name of the Study Programme, name(s) of the author(s),
% student ID number, name of the supervisor, name(s) of the co-supervisor(s) (if any), academic year.
% \\
% Be sure to select a title that is meaningful.
% It should contain important keywords to be identified by indexer.
% Keep the title as concise as possible and comprehensible even to people who are not experts in your field.
% The title has to be chosen at the end of your work so that it accurately captures the main subject of the manuscript.

% It is convenient to break the article format of your thesis (in article format) into sections and subsections. 
% If necessary, subsubsections, paragraphs and subparagraphs can be used. 
% A new section is created by the command
% \begin{verbatim}
% \section{Title of the section}
% \end{verbatim}
% The numbering can be turned off by using \verb|\section*{}|.
% A new subsection is created by the command
% \begin{verbatim}
% \subsection{Title of the subsection}
% \end{verbatim}
% and, similarly, the numbering can be turned off by adding an asterisk as follows 
% \begin{verbatim}
% \subsection*{}
% \end{verbatim}
% It is recommended to give a label to each section by using the command
% \begin{verbatim}
% \label{sec:section_name}%
% \end{verbatim}
% where the argument is just a text string that you'll use to reference that part
% as follows: \textit{Section~\ref{sec:introduction} contains \sc{INTRODUCTION}  \dots}.

% lorem ipsum
\lipsum[3-7]