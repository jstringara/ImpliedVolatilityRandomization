\section{Literature Review and Analysis}

\subsection{Problem Setting}
In financial markets, the pricing of vanilla european options (either calls or puts) is a fundamental task that relies on the accurate estimation of the underlying asset's volatility.
Customarily, what markets quote is not directly the price of such options, but rather the \textbf{Black and Scholes implied volatility} surface $\hat{\sigma}(T,K)$, which is a function of time to maturity $T$ and strike price $K$.
Therefore given a certain combination of time to maturity and strike price, the market will provide us with a value $\hat{\sigma}(T,K)$ that is the implied volatility of the option with such parameters such that the option's price can be expressed as:
\begin{equation}
    V_c(T,K) = BS(t_0, S_0, T, K; \hat{\sigma}(T,K)),
\end{equation}

where $BS(t_0, S_0, T, K; \hat{\sigma}(T,K))$ is the Black-Scholes price of a European call option with strike $K$ and maturity $T$, evaluated at time $t_0$ with underlying asset price $S_0$ and implied volatility $\hat{\sigma}(T,K)$ and a constant risk-free interest rate $r$.
As is well know, the Black-Scholes model assumes that the underlying asset price follows a geometric Brownian motion, which is unfortunately is known to be a poor approximation of real market dynamics.
Usually we will find that the market provides us with a full set of quotes for a grid of maturities and strikes, which we can denote as:
\begin{equation}
    \Theta_{mkt} = \{\hat{\sigma}_{mkt}(T_n, K_m) : 1 \leq n \leq N, 1 \leq m \leq M\},
\end{equation}

Unfortunately, usually market quotes are rather sparse, and we will not have a full grid of quotes for all possible (or indeed for any useful combination of) maturities and strikes.
This is especially true for markets with low liquidity, where we will have only a few quotes for each maturity and strike.
This is a major problem since many financial applications require market makers to to be able to provide quotes for option combinations that are not directly quoted by the market (e.g. the combination $(T,K)$ for which we do not have a market quote).
Therefore the problem of constructing a \textbf{clean implied volatility surface} $\hat{\sigma}(T,K)$ from the discrete and noisy market quotes $\Theta_{mkt}$ is a fundamental task in quantitative finance.
More generally it can be said that the main problem is to extend the discrete market quotes $\Theta_{mkt}$ to a continuous (or even smooth) function $\hat{\sigma}(T,K): \Pi \rightarrow \mathbb{R}_+$ that provides implied volatility values for any desired combination of time to expiry and strike on domain $\Pi \subset (t_0, \infty) \times \mathbb{R}_+$.

As we have seen such an extension is not trivial, since two key properties must be satisfied by the resulting surface $\hat{\sigma}(T,K)$:

\begin{itemize}
    \item \textbf{Consistency with market quotes}: The surface must match the market quotes $\hat{\sigma}_{mkt}(T_n, K_m)$ at the given maturities and strikes, meaning that for any $(T_n, K_m) \in \Theta_{mkt}$ we must have $\hat{\sigma}(T_n, K_m) = \hat{\sigma}_{mkt}(T_n, K_m)$.
    \item \textbf{Arbitrage-free}: The surface must be free of static arbitrage opportunities, which means that the implied volatility must be consistent with the no-arbitrage pricing theory.
\end{itemize}

If and when such a surface is constructed, it can be then used to price a wide plethora of exotic derivatives, recreate the probability distribution of the underlying asset $S$, calibrate pricing models for exotic options or general trading purposes by assessing volatility expectations, arbitrage opportunities or for market makers to set the correct pricing of novel financial instruments.

\subsection{Existing Literature}

In recent decades, the modeling of implied volatility surfaces has emerged as a central problem in quantitative finance, due to its foundational role in derivative pricing, hedging, and risk management.
The construction of clean implied volatility surfaces from discrete and noisy market option quotes is fundamental for pricing exotic derivatives, setting margin requirements, and enabling competitive market-making strategies.
Traditional models, such as the Stochastic Volatility Inspired (SVI) and Stochastic Alpha Beta Rho (SABR) parametrizations, provide efficient and arbitrage-free representations of market-implied volatilities.
However, these models often struggle to replicate the nuanced structures observed in real market conditions, especially under short maturities or near market events like earnings announcements.
The essence of constructing an implied volatility surface is to encode discrete market quotes $\Theta_{mkt} = \{\hat{\sigma}_{mkt}(T_n, K_m) : 1 \leq n \leq N, 1 \leq m \leq M\}$ into a continuous function $\hat{\sigma}(T,K): \Pi \rightarrow \mathbb{R}_+$ that provides implied volatility values for any desired combination of time to expiry and strike on domain $\Pi \subset (t_0, \infty) \times \mathbb{R}_+$.
The primary challenges are ensuring that the surface extends the discrete quotes while remaining free of static arbitrage opportunities.

In \textit{Volatility Parametrizations with Random Coefficients}, Zaugg, Perotti, and Grzelak (2024) \cite{zaugg2024volatility} propose a generalizable and analytically tractable extension to classical parametric volatility models through the \textbf{randomization of model parameters}.
Their methodology addresses a critical gap in existing literature: the structural rigidity imposed by fixed-parameter surfaces that often leads to poor calibration or unrealistic parameter estimates when market conditions fall outside the parametrization's scope.
Indeed classic models like SVI and SABR, while efficient, are limited in their ability to adapt to irregular market conditions due to their reliance on a family of curves defined by the parameter space.

\subsection{Parametric vs. Interpolation Approaches}

The literature presents two primary methodologies for constructing implied volatility surfaces: interpolation schemes and parametric approaches.

\textbf{Interpolation Methods}: This is the simplest approach, it involves direct interpolation and extrapolation of available volatility quotes.
However, as demonstrated by Fengler (2005), such methods are generally not arbitrage-free since direct interpolation on the volatility surface often leads to non-convex option pricing functions. This mispricing is particularly problematic for assets with few available quotes, where interpolation determines a large portion of the surface. Arbitrage-free interpolation is feasible on option prices rather than implied volatilities, but requires root-finding algorithms to recover implied volatilities and remains challenging for illiquid markets.

\textbf{Parametric Approaches}: These methods define a family of smooth surfaces $\hat{\sigma}(T,K;\mathbf{p})$ parametrized by vector $\mathbf{p} \in \mathbb{R}^m$, with arbitrage-freeness guaranteed through parameter constraints. The surface is calibrated by minimizing differences to market quotes $\Theta_{mkt}$. While exact matching of market quotes is sacrificed, parametric methods offer several advantages: guaranteed arbitrage-freeness, interpretable parameters, and ability to generate surfaces from minimal market data.

\subsection{Parametric Volatility Surfaces and Their Limitations}

Let $\hat{\sigma}(T, K; \mathbf{p})$ denote a classical implied volatility surface parametrized by a vector $\mathbf{p} \in \mathbb{R}^m$, defined over a domain $\Pi \subseteq (t_0, \infty) \times \mathbb{R}_+$.
Classical models such as SVI and SABR define such surfaces based on analytically tractable stochastic processes.
These parametrizations are valued for their:

\begin{itemize}
    \item \textit{Arbitrage-free constraints}: ensured through structural properties of the surface or well-behaved parameter spaces;
    \item \textit{Calibration efficiency}: due to closed-form or semi-closed-form pricing equations;
    \item \textit{Interpretability}: model parameters have intuitive effects on smile, skew, and term structure;
    \item \textit{Robustness}: ability to generate surfaces from limited market data without requiring minimum quote thresholds.
\end{itemize}

However, a major drawback arises when the true market surface lies outside the span of $\{\hat{\sigma}(T,K; \mathbf{p})\}_{\mathbf{p} \in D}$, resulting in model misspecification and poor market fit.
This limitation manifests in two critical scenarios:

\textbf{Systematic Limitations}: Certain market behaviors are fundamentally incompatible with classical parametrizations. For instance, short-maturity options in equity markets exhibit steeper at-the-money implied volatility term structures than those attainable under regular stochastic volatility models. This reflects the inability of single-factor diffusion models to capture the complex dynamics of near-expiry options.

\textbf{Event-Driven Irregularities}: Markets can exhibit irregular behaviors during periods of heightened uncertainty. Very short-term options markets near earnings announcements commonly display W-shaped volatility curves or "mustache" shapes. These patterns arise from bimodal risk-neutral probability density functions reflecting the dichotomous nature of earnings outcomes. Classical diffusion models cannot replicate such multi-modal distributions due to their reliance on continuous diffusive drivers.

\subsection{Challenges in Short-Term Options Modeling}

The modeling of short-maturity options presents particular challenges that expose the limitations of traditional parametric approaches. Near-expiry options are characterized by:

\begin{itemize}
    \item \textit{Heightened sensitivity to discrete events}: Earnings announcements, regulatory decisions, or other binary outcomes can create bimodal risk-neutral densities.
    \item \textit{Extreme volatility shapes}: W-shaped implied volatility curves that cannot be captured by standard one-factor models.
    \item \textit{Rapid parameter changes}: Traditional calibration procedures may fail or yield extreme parameter values indicating model inconsistency.
    \item \textit{Market microstructure effects}: Bid-ask spreads and liquidity constraints can distort implied volatility patterns.
\end{itemize}

These challenges highlight the industry's need for more flexible parametrizations that can adapt to irregular market conditions while preserving computational efficiency and arbitrage-freeness.

\subsection{Randomization Framework}

The authors propose enhancing the flexibility of $\hat{\sigma}(T,K; \mathbf{p})$ by replacing one or more parameters $p_i$ with a random variable $\vartheta$. The resulting implied volatility surface is derived from the expectation of the corresponding pricing function:

\[
    \hat{\sigma}(T,K) \text{ such that } BS(t_0, S_0, T, K; \hat{\sigma}(T,K)) = \mathbb{E}_{\vartheta}[V_c(T,K; p(\vartheta))].
\]

This yields a mixture model of option prices that remains arbitrage-free under mild regularity assumptions on $\vartheta$ and $\hat{\sigma}$. The authors show that the implied pricing function preserves convexity and monotonicity (conditions for butterfly and calendar spread arbitrage, respectively) through integral arguments and convex combinations of valid surfaces.

The randomization framework draws inspiration from the RAnD (Risk and Drift) method while providing a more general and analytically tractable approach. By treating parameters as random variables rather than fixed constants, the method enables classical parametrizations to capture market behaviors previously outside their scope.

\subsection{Discretization and Parametric Extension}

To make this construction computationally feasible, the paper employs Gaussian quadrature to discretize the mixture:

\[
    V_c(T,K) \approx \sum_{n=1}^{N_q} \lambda_n V_c(T,K; p(\theta_n)).
\]

This transforms the randomized surface into an extended parametric form:
\[
    \hat{\sigma}(T,K; \mathbf{p}^*) = \text{BS}^{-1}\left(\sum_{n=1}^{N_q} \lambda_n V_c(T,K; p(\theta_n))\right),
\]
where $\mathbf{p}^*$ includes the original parameters (excluding $p_i$) and the distributional parameters of $\vartheta$ (e.g., $(\mu, \nu)$ for lognormal or $(k, \theta)$ for Gamma). The resulting model is therefore not only arbitrage-free but also retains parametric tractability with enhanced expressiveness.

\subsection{Analytic Expansion of Implied Volatility}

To avoid repeated root-finding, the authors derive a Taylor expansion of the randomized implied volatility function in terms of log-moneyness $m = \log(S_0/K) + rT$. Extending the approach of Brigo and Mercurio (2002), they express the implied volatility $\hat{\sigma}(T,K)$ as:

\[
    \hat{\sigma}(T,K) \approx P(0) + \frac{P^{(2)}(0)}{2!}m^2 + \frac{P^{(4)}(0)}{4!}m^4 + \frac{P^{(6)}(0)}{6!}m^6 + \mathcal{O}(m^8),
\]

where the coefficients $P^{(2i)}(0)$ are computed using the mixture weights and base volatilities. This expansion enables fast evaluation across large strike/time grids and is validated against root-finding solutions.

\subsection{Applications and Empirical Evidence}

\paragraph{1. Randomized Flat Volatility:}
A trivial constant-volatility model, when randomized with a lognormal variable, can produce volatility smiles similar to real market shapes. The resulting model is parsimonious (only two parameters), efficient, and retains analytical tractability.

\paragraph{2. Randomized SABR:}
By randomizing the volatility-of-volatility parameter $\gamma$ in the SABR model using a Gamma distribution, the authors significantly improve the model's ability to fit short-maturity SPX options. The randomized model outperforms standard SABR in terms of sum and mean squared errors, and captures skew curvature more accurately, particularly for options with expiry dates of less than six months.

\paragraph{3. Spot Price Randomization:}
To model W-shaped or bimodal implied volatilities (as observed before earnings announcements), the authors propose randomizing the spot price $S_0$. This produces a mixture of shifted probability density functions and enables modeling of multi-modal risk-neutral densities. The resulting surfaces are shown to be arbitrage-free as long as the random spot is centered at $S_0$. This approach proves particularly effective for modeling near-maturity options preceding earnings announcements, where traditional parametrizations fail completely.

\subsection{Position in Literature}

This framework complements and extends several strands of the literature:

\begin{itemize}
    \item \textit{Parametric Models}: Offers more flexibility than SABR or SVI while retaining analytical structure and computational efficiency.
    \item \textit{Mixture Models}: Provides a cleaner, parsimonious alternative to lognormal mixtures with fewer parameters and guaranteed arbitrage-freeness.
    \item \textit{Machine Learning}: Maintains interpretability and arbitrage-freeness, unlike black-box neural network approaches that often struggle with financial constraints.
    \item \textit{Market-Making Applications}: Enables practitioners to price options in extreme market conditions where traditional models fail, particularly important for liquidity providers in volatile markets.
\end{itemize}

The methodology addresses the fundamental trade-off between model flexibility and computational tractability that has long challenged practitioners. Unlike purely statistical or machine learning approaches, the randomization framework preserves the theoretical foundations of parametric models while extending their practical applicability.

\subsection{Conclusion}

The methodology introduced by Zaugg et al.\ strikes a balance between flexibility, computational efficiency, and financial interpretability. It enables classical models to adapt to irregular market conditions through controlled randomness, while preserving arbitrage constraints and maintaining calibration tractability.

The framework is particularly valuable for addressing the limitations of parametric models in extreme market conditions, such as pre-earnings volatility spikes and short-maturity option anomalies. By providing a generic enhancement method, it offers practitioners a systematic approach to improving existing parametrizations without abandoning their theoretical foundations.

From a research and practical standpoint, the framework opens further lines of inquiry:
\begin{itemize}
    \item Exploring multi-parameter or distributional combinations for enhanced flexibility.
    \item Developing adaptive calibration strategies for highly dynamic markets.
    \item Applying the method to exotic options or illiquid asset classes where traditional methods struggle.
    \item Investigating the framework's performance during extreme market events and crisis periods.
\end{itemize}

For the remainder of this thesis, we will build on this foundation to implement, analyze, and extend randomized volatility parametrizations using empirical and synthetic data, with particular focus on the challenging scenarios where traditional models reach their limits.
