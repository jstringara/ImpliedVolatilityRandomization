\section{Literature Review and Analysis}

\subsection{Problem Setting}

In financial markets, the pricing of vanilla European options, either calls or puts, is a fundamental task that hinges on the accurate estimation of the underlying asset's volatility.
In practice, markets do not quote option prices directly, but rather the \textbf{Black-Scholes implied volatility}, denoted by $\hat{\sigma}(T, K)$, which depends on the option's time to maturity $T$ and strike price $K$.
Given a particular combination of $(T, K)$, the market provides a value $\hat{\sigma}(T, K)$ such that the corresponding option price can be computed using the Black-Scholes formula:
\begin{equation}
    V_c(T,K) = BS(t_0, S_0, T, K; \hat{\sigma}(T,K)),
\end{equation}

where $BS(t_0, S_0, T, K; \hat{\sigma}(T,K))$ denotes the Black-Scholes price of a European call option with strike $K$ and maturity $T$, evaluated at time $t_0$, given a spot price $S_0$, a constant risk-free rate $r$, and implied volatility $\hat{\sigma}(T, K)$.

As is well known, the Black-Scholes model assumes that the underlying asset follows a geometric Brownian motion with constant volatility, a convenient but overly simplistic assumption that fails to reflect many features observed in real markets. Nevertheless, implied volatility serves as a standard market quoting convention, and practitioners work within this framework to price and hedge derivatives.

Typically, market data consists of implied volatility quotes for a finite grid of strikes and maturities. We denote this discrete dataset as:

\begin{equation}
    \Theta_{\text{mkt}} = \left\{ \hat{\sigma}_{\text{mkt}}(T_n, K_m) : 1 \leq n \leq N,\ 1 \leq m \leq M \right\}.
\end{equation}

In liquid markets, this set may be fairly dense, but in many cases, especially for less liquid underlyings, only a sparse set of quotes may be observed, with limited coverage in both strike and maturity.
This sparsity presents a major obstacle; in most real-world applications, one needs to produce implied volatility values for strike-maturity pairs that may not be directly quoted by the market.
For example, a market maker may need to quote an option with a non-standard expiry, or a trader may need to evaluate the price of a structured product referencing off-grid options.
The task, then, is to construct a \textbf{clean implied volatility surface} $\hat{\sigma}(T, K)$ from the discrete and potentially noisy market data $\Theta_{\text{mkt}}$.
More precisely, we seek a function
\[
    \hat{\sigma}(T, K): \Pi \to \mathbb{R}_+,
\]
defined over a suitable domain $\Pi \subset (t_0, \infty) \times \mathbb{R}_+$, such that the resulting surface provides meaningful volatility values for any desired combination of time to maturity and strike price.

This problem is far from trivial, as the resulting surface $\hat{\sigma}(T, K)$ must satisfy two essential properties:

\begin{itemize}
    \item \textbf{Consistency with market data}: the surface should match the observed market quotes at the points where they are available.
          That is, for any $(T_n, K_m) \in \Theta_{\text{mkt}}$, we must find that $\hat{\sigma}(T_n, K_m) = \hat{\sigma}_{\text{mkt}}(T_n, K_m)$.
    \item \textbf{Absence of arbitrage}: the surface must not allow for arbitrage opportunities when plugged into the Black-Scholes formula.
          This means that the resulting option prices must respect key no-arbitrage conditions, such as monotonicity and convexity in strike, and the call-put calendar spread inequalities in maturity.
\end{itemize}

If such a surface can be constructed, one that is both consistent with the market and arbitrage-free, it becomes a powerful tool.
It allows for pricing of exotic derivatives, reconstruction of the risk-neutral distribution of the underlying asset, model calibration, and the generation of synthetic quotes.
Ultimately, it enables market participants to respond to pricing and risk management tasks with greater confidence and consistency, even in the absence of direct market observations.

\subsection{Existing Literature}

Implied volatility surface modeling has been extensively studied using stochastic and local volatility models, statistical analyses, arbitrage-aware interpolation schemes, and, more recently, machine learning techniques.
A comprehensive review is provided by Homescu~\cite{homescu2011implied}.
Here, we summarize and review the approaches most relevant to our methodology.

\subsubsection{Interpolation and Extrapolation Schemes}

A common and straightforward method to construct an implied volatility surface is to interpolate and extrapolate the available market quotes, yielding a surface $\hat\sigma: (T,K) \mapsto \mathbb{R}_+$.
However, interpolating directly in the implied volatilities space is known to produce surfaces that may violate arbitrage constraints, particularly the convexity of option prices with respect to the strike~\cite{andreasen2011volatility}.
This issue becomes particularly problematic when the dataset is sparse, as the interpolation then determines much of the surface beyond the observed points, leading to financially implausible shapes.

To address this, several methods construct arbitrage-free surfaces by interpolating on option prices instead of implied volatilities~\cite{lefloch2021arbitrage, corbetta2019robust}.
These approaches ensure consistency with no-arbitrage principles, but they require a numerical inversion, via root finding algorithms, of the Black-Scholes formula to recover implied volatilities, since the mapping from price to volatility is implicit.
Interpolation schemes are attractive because they exactly match the observed market quotes $\Theta_{\text{mkt}}$ by construction.
However, they often suffer from poor extrapolation behavior, especially when quotes are irregularly distributed or sparse~\cite{guo2016generalized}.
Furthermore, interpolated surfaces may lack the smoothness and robustness required for risk management or model calibration.

\subsubsection{Parametric Volatility Surfaces}

An alternative to interpolation is to define a family of parametric implied volatility surfaces indexed by a low-dimensional vector of parameters $\mathbf{p} \in \mathbb{R}^m$.
This defines a family of smooth, arbitrage-free surfaces specified by a function:
\[
    \hat{\sigma}(T, K; \mathbf{p}): \Pi \to \mathbb{R}_+,
\]
where the surface is fit to market quotes by solving a calibration problem:
\[
    \mathbf{p}^* = \arg\min_{\mathbf{p}} \sum_{(T_n, K_m) \in \Theta_{\text{mkt}}} \| \hat{\sigma}(T_n, K_m; \mathbf{p}) - \hat{\sigma}_{\text{mkt}}(T_n, K_m) \|,
\]
for a suitable choice of norm $\|\cdot\|$.
The ability to fit the market, therefore, directly depends on the “size” of the set of functions defined by $\{\hat{\sigma}(T, K; \mathbf{p})\}_{\mathbf{p} \in D}$, where $D \subseteq \mathbb{R}^m$ is the parameter space.
Popular examples of such models include the Stochastic Volatility Inspired (SVI) parametrization introduced by Gatheral~\cite{gatheral2011volatility} and further refined into an arbitrage-free version with Jacquier~\cite{gatheral2014arbitrage}, as well as the SABR model of Hagan et al.~\cite{hagan2002managing}.
These models are widely adopted due to their analytical tractability, ease of calibration, and robustness in sparse data regimes.

A key advantage of parametric surfaces is their ability to enforce absence of arbitrage through structural constraints on the parameter space $\mathbf{p} \in D \subseteq \mathbb{R}^m$.
In many cases, the models are derived from an underlying stochastic process for the asset, ensuring that arbitrage-free conditions are satisfied for any admissible parameter choice.
For instance, both SVI and SABR offer closed-form or semi-analytical expressions for implied volatility, and their parameters directly reflect key features of the smile and term structure.

This makes parametric models particularly appealing: they offer interpretability (e.g., linking parameters to skew or curvature), do not require a large number of quotes to produce a valid surface, and inherently preserve arbitrage-free properties.

However, the main limitation of classical parametrizations is their lack of flexibility.
When the true market-implied surface falls outside the functional class defined by the model, even the best-fitting parameter vector $\mathbf{p}^*$ may result in a poor approximation.
This is particularly relevant for short-dated equity options near corporate events, such as earnings announcements, where implied volatility smiles may become ``W-shaped'' or ``mustache-shaped'', indicating bimodal risk-neutral densities.
Such phenomena are well-documented by Glasserman and Pirjol~\cite{glasserman2023w} and Alexiou et al.~\cite{alexiou2023pricing}, and cannot be captured by standard diffusion-based models.

\subsubsection{Randomized Parametrizations}

To overcome the expressive limitations of classical models, recent work has introduced \textbf{randomized parametrizations}.
Zaugg, Perotti, and Grzelak~\cite{zaugg2024volatility} propose a framework in which one or more deterministic parameters in a standard model, such as SABR, are replaced by random variables.
The resulting implied volatility surface is then obtained by integrating over the distribution of these parameters, yielding enhanced flexibility while preserving analytical structure and arbitrage-freeness.

This randomized approach effectively enlarges the space of volatility surfaces without increasing the dimensionality of the calibration problem.
By specifying the distributional form of the random parameter(s) using a small set of additional hyperparameters, the authors retain tractability and control over the shape of the surface.
In particular, they demonstrate strong calibration performance for short-dated SPX optIions (with maturities below six months), a region where traditional parametrizations often struggle.
In practice we are effectively replacing a single deterministic volatility surface with a mixture of surfaces, each corresponding to a different realization of the random parameter(s).

Moreover, Zaugg et al.\ also consider \textbf{spot price randomization}, a novel mechanism that models uncertainty in the initial asset value.
This is especially relevant for near-expiry options subject to potential regime changes, such as equities approaching earnings announcements.
In such scenarios, the implied volatility surface can display multi-modality that reflects the market's anticipation of discrete jumps in price.
Spot randomization captures this behavior effectively, while ensuring the resulting surface remains arbitrage-free.

These innovations offer a general and modular methodology to enrich existing volatility surface models.
The approach is empirically validated through calibration examples on S\&P 500 and AMZN options, where the randomized models deliver better fits than their classical counterparts, especially in challenging market regimes.
