% Set the path for images
\graphicspath{{images/}}

% Insert here the info that will be displayed into your Title page 
% -> title of your work
\renewcommand{\title}{Title}
% -> author name and surname
\renewcommand{\author}{Jacopo Stringara}
% -> MSc course
\newcommand{\course}{Mathematical Engineering - Quantitative Finance}
% -> advisor name and surname
\newcommand{\advisor}{Prof. Daniele Marazzina}
% IF AND ONLY IF you need to modify the co-supervisors you also have to modify the file config/title_page.tex (ONLY where it is marked)
% \newcommand{\firstcoadvisor}{Name Surname} % insert if any otherwise comment
% \newcommand{\secondcoadvisor}{Name Surname} % insert if any otherwise comment
% -> author ID
\newcommand{\ID}{10687726}
% -> academic year
\newcommand{\YEAR}{2024-2025}
% -> abstract (only in English)
\renewcommand{\abstract}{Here goes the Abstract in English of your thesis (in article format)
followed by a list of keywords.
The Abstract is a concise summary of the content of the thesis (single page of text)
and a guide to the most important contributions included in your thesis.
The Abstract is the very last thing you write.
It should be a self-contained text and should be clear
to someone who hasn't (yet) read the whole manuscript.
The Abstract should contain the answers to the main research questions
that have been addressed in your thesis.
It needs to summarize the motivations and the adopted approach as well as
the findings of your work and their relevance and impact.
The Abstract is the part appearing in the record of your thesis inside POLITesi,
the Digital Archive of PhD and Master Theses (Laurea Magistrale) of Politecnico di Milano.
The Abstract will be followed by a list of four to six keywords.
Keywords are a tool to help indexers and search engines to find relevant documents.
To be relevant and effective, keywords must be chosen carefully.
They should represent the content of your work and be specific to your field or sub-field.
Keywords may be a single word or two to four words. }

% -> key-words (only in English)
\newcommand{\keywords}{here, the keywords, of your thesis}
